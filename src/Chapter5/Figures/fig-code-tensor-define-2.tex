%%%------------------------------------------------------------------------------------------------------------
\begin{tcolorbox}[enhanced,width=12cm,frame engine=empty,boxrule=0.1mm,size=title,colback=blue!10!white]
\begin{flushleft}
{\scriptsize
\begin{tabbing}
\texttt{XTensor tensor;} \hspace{14em} \= // 声明张量tensor \\
\texttt{int sizes[6] = \{2,3,4,2,3,4\};} \> // 张量的形状为2*3*4*2*3*4 \\
\texttt{InitTensor(\&tensor, 6, sizes, X\_FLOAT);} \> // 定义形状为sizes的6阶张量
\end{tabbing}
}
\end{flushleft}
\end{tcolorbox}
\hspace{0.1in} \scriptsize{(a) NiuTensor定义张量程序}
\\
\begin{tcolorbox}[enhanced,width=12cm,frame engine=empty,boxrule=0.1mm,size=title,colback=blue!10!white]
\begin{flushleft}
{\scriptsize
\begin{tabbing}
\texttt{XTensor a, b, c;} \hspace{13.5em} \= // 声明张量tensor \\
\texttt{InitTensor1D(\&a, 10, X\_INT);} \> // 10维的整数型向量\\
\texttt{InitTensor1D(\&b, 10);} \> // 10维的向量,缺省类型(浮点)\\
\texttt{InitTensor4D(\&c, 10, 20, 30, 40);} \> // 10*20*30*40的4阶张量(浮点)
\end{tabbing}
}
\end{flushleft}
\end{tcolorbox}
\hspace{0.1in} \scriptsize{(b) 定义张量的简便方式程序}
\\
\begin{tcolorbox}[enhanced,width=12cm,frame engine=empty,boxrule=0.1mm,size=title,colback=blue!10!white]
\begin{flushleft}
{\scriptsize
\begin{tabbing}
\texttt{XTensor tensorGPU;} \hspace{12.5em} \= // 声明张量tensor \\
\texttt{InitTensor2D(\&tensorGPU, 10, 20,} $\backslash$ \> // 在编号为0的GPU上定义张量 \\
\hspace{6.7em} \texttt{X\_FLOAT, 0);}
\end{tabbing}
}
\end{flushleft}
\end{tcolorbox}
\hspace{0.1in} \scriptsize{(c) 在GPU上定义张量程序}
%%%------------------------------------------------------------------------------------------------------------

